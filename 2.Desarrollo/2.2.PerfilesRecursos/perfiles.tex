Administrar los perfiles de límite de recursos en una base de datos es una buena práctica para asegurar un uso eficiente y controlado de los recursos por parte de los usuarios. Esto es especialmente útil en bases de datos \textbf{\emph{Oracle}}, en las cuales se pueden definir perfiles que restringen ciertos recursos como el uso de CPU, la cantidad de sesiones abiertas, el tiempo de inactividad, y mucho más.

\begin{enumerate}
  \item \textbf{¿Qué es un perfil de límite de recursos?}
  
  Un \textbf{perfil de base de datos} permite definir límites sobre el uso de recursos del sistema para los usuarios. Se puede asignar un perfil a un usuario y así controlar:

  \begin{itemize}
    \item Uso de CPU y tiempo de sesión.
    \item Conexiones concurrentes.
    \item Políticas de contraseñas y bloqueo de cuentas.
  \end{itemize}

  \item \textbf{Activar la administración de recursos en Oracle}
  
  \begin{lstlisting}
    ALTER SYSTEM SET RESOURCE_LIMIT = TRUE;
  \end{lstlisting}

  \item \textbf{Crear un perfil personalizado}
  
  El siguiente ejemplo crea un perfil llamado \textbf{\emph{mi{\textunderscore}perfil}} que establece límites en el uso de recursos:

  \begin{lstlisting}
    CREATE PROFILE mi_perfil LIMIT
    SESSIONS_PER_USER         2                  -- Maximo 2 sesiones por usuario
    CPU_PER_SESSION           10000              -- Maximo 10,000 unidades de CPU por sesion
    CPU_PER_CALL              1000               -- Maximo 1,000 unidades de CPU por llamada
    CONNECT_TIME              60                 -- Maximo 60 minutos de conexion
    IDLE_TIME                 10                 -- Tiempo inactivo de 10 minutos antes de desconectar
    FAILED_LOGIN_ATTEMPTS     5                  -- Bloqueo despurs de 5 intentos fallidos
    PASSWORD_LIFE_TIME        30                 -- La contrasena expira en 30 días
    PASSWORD_REUSE_TIME       365                -- Reutilizacion de contrasena despues de 1 ano
    PASSWORD_REUSE_MAX        5                  -- Reutilizar contrasena despues de 5 cambios;
  \end{lstlisting}

  \item \textbf{Asignar el perfil a un usuario}
  
  \begin{lstlisting}
    ALTER USER usuario1 PROFILE mi_perfil;
  \end{lstlisting}

  Esto significa que el \textbf{\emph{usuario1}} estará sujeto a las restricciones definidas en \textbf{\emph{mi{\textunderscore}perfil}}.

  \item \textbf{Modificar un perfil existente}
  
  \begin{lstlisting}
    ALTER PROFILE mi_perfil LIMIT
    IDLE_TIME 15;  -- Aumentar el tiempo inactivo a 15 minutos
  \end{lstlisting}

  \item \textbf{Eliminar un perfil}
  
  \begin{lstlisting}
    DROP PROFILE mi_perfil CASCADE;
  \end{lstlisting}

  El uso de \textbf{\emph{CASCADE}} asegura que el perfil se elimine incluso si está asignado a usuarios.

  \item \textbf{Consultar información sobre perfiles}
  
  \begin{lstlisting}
    SELECT PROFILE, RESOURCE_NAME, LIMIT 
    FROM DBA_PROFILES
    WHERE PROFILE = 'MI_PERFIL';
  \end{lstlisting}
\end{enumerate}