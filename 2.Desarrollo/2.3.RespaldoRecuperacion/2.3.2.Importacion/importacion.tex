La importación es el proceso de restaurar los datos y objetos desde un archivo de exportación previamente creado, en un sitema como \textbf{Oracle Database} esto se puede lograr utilizando textbf{\emph{impdp}}.

\begin{enumerate}
  \item \textbf{Importar un esquema completo}
  
  Este comando importa todos los objetos desde un archivo \textbf{\emph{.dmp}} en el esquema \textbf{\emph{mi{\textunderscore}usuario}}.

  \begin{lstlisting}
    impdp mi_usuario/mi_contraseña DIRECTORY=mi_directorio DUMPFILE=backup_esquema.dmp SCHEMAS=mi_usuario LOGFILE=import_esquema.log
  \end{lstlisting}

  \item \textbf{Importar tablas específicas}
  
  Restaurar algunas tablas:

  \begin{lstlisting}
    impdp mi_usuario/mi_contraseña DIRECTORY=mi_directorio DUMPFILE=backup_tablas.dmp TABLES=tabla1,tabla2 LOGFILE=import_tablas.log
  \end{lstlisting}

  \item \textbf{Importar con remapeo de esquema}
  
  Para importar datos a un esquema diferente, puedes usar 
  
  \textbf{\emph{REMAP{\textunderscore}SCHEMA}}:

  \begin{lstlisting}
    impdp mi_usuario/mi_contraseña DIRECTORY=mi_directorio DUMPFILE=backup_esquema.dmp REMAP_SCHEMA=antiguo_usuario:nuevo_usuario LOGFILE=import_remapeo.log
  \end{lstlisting}

  \item \textbf{Importar solo datos (sin índices, constraints, etc.)}
  
  Para importar únicamente los datos sin los objetos adicionales:

  \begin{lstlisting}
    impdp mi_usuario/mi_contraseña DIRECTORY=mi_directorio DUMPFILE=backup_tablas.dmp CONTENT=DATA_ONLY LOGFILE=import_datos.log
  \end{lstlisting}
\end{enumerate}