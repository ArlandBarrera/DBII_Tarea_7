En muchas ocasiones resulta necesario limitar el espacio de almacenamiento en un servidor atendiendo a diversos criterios, por ejemplo la limitación del espacio disponible para correo electrónico o para albergue de páginas web.

Para limitar las cuotas de espacios de tablas en una base de datos, se pueden utilizar varias estrategias dependiendo del sistema de gestión de bases de datos (SGBD) utilizado. A continuación, se presentan algunas opciones para MySQL y SQL Server:

\textbf{MySQL}

\begin{enumerate}
  \item \textbf{Asignación de cuotas a una base de datos existente:} Utilizar la sentencia \textbf{\emph{CREATE TABLESPACE}} con la opción \textbf{\emph{QUOTA}} para establecer un límite de espacio para una base de datos específica. Por ejemplo:
  
  \begin{lstlisting}
    CREATE TABLESPACE mydb QUOTA 100M;
  \end{lstlisting}
  
  Esto establece un límite de 100 MB para la base de datos \textbf{\emph{mydb}}.

  \item \textbf{Asignación de cuotas a un usuario:} Utilizar la sentencia \textbf{\emph{CREATE USER}} con la opción \textbf{\emph{QUOTA}} para establecer un límite de espacio para un usuario específico. Por ejemplo:

  \begin{lstlisting}
    CREATE USER myuser IDENTIFIED BY `mypassword' QUOTA 50M ON mydb;
  \end{lstlisting}

  Esto establece un límite de 50 MB para el usuario \textbf{\emph{myuser}} en la base de datos \textbf{\emph{mydb}}.

\end{enumerate}

\textbf{SQL SERVER}

\begin{enumerate}
  \item \textbf{Asignación de cuotas a un usuario:} Utilizar la sentencia \textbf{\emph{CREATE USER}} con la opción \textbf{\emph{QUOTA}} para establecer un límite de espacio para un usuario específico. Por ejemplo:

  \begin{lstlisting}
    CREATE USER myuser FOR LOGIN = 'mylogin' DEFAULT_TABLESPACE = 'mydb' QUOTA 100MB;
  \end{lstlisting}

  Esto establece un límite de 100 MB para el usuario \textbf{\emph{myuser}} en la base de datos \textbf{\emph{mydb}}.

  \item \textbf{Configuración de cuotas en el nivel de instancia:} En la configuración de la instancia de SQL Server, se puede establecer un límite de espacio global para todas las bases de datos. Para hacer esto, se necesita el permiso \textbf{\emph{ALTER INSTANCE}} y se utiliza la sentencia \textbf{\emph{ALTER INSTANCE}} con la opción \textbf{\emph{SET MAX{\textunderscore}DATA{\textunderscore}SIZE}}. Por ejemplo:
  
  \begin{lstlisting}
    ALTER INSTANCE SET MAX_DATA_SIZE = 500GB;
  \end{lstlisting}

  Esto establece un límite de 500 GB para el tamaño total de todas las bases de datos en la instancia de SQL Server.
\end{enumerate}